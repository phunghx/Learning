\documentclass{article}
\usepackage[utf8]{inputenc}
 
\title{Business English: Making Presentations}
\author{Huynh Xuan Phung - Coursera}
\date{ }
\usepackage{color}   %May be necessary if you want to color links
\usepackage{hyperref}
\hypersetup{
    colorlinks=true, %set true if you want colored links
    linktoc=all,     %set to all if you want both sections and subsections linked
    linkcolor=blue,  %choose some color if you want links to stand out
} 
\begin{document}
 
\maketitle
 
\tableofcontents
\section{Giving Effective Presentation}
\subsection{Control Nervous and appear more relaxed}

\begin{itemize}
\item{Prepare and Practice: Plan ahead of time}
\item{Calm your Nerves: breath deeply, breathe out longer than you breathe in (count 2 for breath in, 6 for breath out)}
\end{itemize}

\subsection{Posture}
Eye Contact: face the audience not the screen. find a few people to look at

Voice: volume, tempo: don't speak too quickly

Emphasis: chunking, stress some key words. Pause before and after the keywords

Intonation: question or not

Write important words in presentation and check the pronunciation.

Abbreviation: use stand for or is a short for

Stress the final letter in an abbreviation where can say the letters: IT: stress on T (the end of)

Fillers: here's an example, Let me see, UHMM (don't use)

\subsection{Effective Introductions}

Basic plan for a Presentation:
\begin{itemize}
\item{1. Tell people what you plan to say}
\item{2. Say it}
\item{3. Tell them what you have said}
\end{itemize}

Greeting: Good Morning

Topic: Today, I'm going to be talking about .../ In this presentation, we're going to show you ...

Effective Introductions (first 5 minutes) 

Motivate your audience to listen to you

Hooks
\begin{itemize}
\item{Ask question: ask people experience: pause people response }
\item{Tell a story: meaningful/ interesting (paint point/problem)}
\item{Give some surprising information (statistic): Did you know that ...}
\end{itemize}

Keeping the Audience's Attention
\begin{itemize}
\item{wait to stop talking}

\end{itemize}

Direction:

Overview of your presentation: Do not say "introduce our plan". You show your plan. Let me show my plan or Here's an overview of my presentation. 

Objective/ Goals
Use short nouns and verbs

\subsection{Guidelines for Presentation}
A: Attention

B: Benefit

C: Credibility ( They believe what you say)

D: Direction

\subsection{Expression for Attracting the Audience's Attention}

\begin{itemize}
\item{My name is ... / I'm ...}
\item{Today, I'm going to be talking about ...}
\item{in this presentation, we're going to show you ...}
\item{How many of you ...?}
\item{Please raise your hands if you ...}
\item{Did you know that ...}
\item{Here's a statistic that may surprise you}
\item{Let me show you our plan}
\item{Here's an overview of ...}
\item{My goal is to ...}
\item{Here's what our objectives are today}
\end{itemize}

\section{Transitions and Conclusion}

D: Direction --- signpost: sequence

To begin, First, Then, Next

Next topic: Let's move on to ... , Now let's look at ..., Moving on to ...

Signposts: Explanations:  So, why does this happen? You may be wondering when/why/how ...

Signposts: showing Significance or Effects
\begin{itemize}
\item{What's the significance of this?}
\item{Why is this important?}
\item{This is important because...}
\end{itemize}

Signpost: Going Back

--- Let's go back, Backing up to

Signposts: Details

---There are two important ... to consider

Signposts: Examples

---For example

Signpost: Rephrasing

--- In other words, What I mean that ...

Signposts: Conclusions

--- So, to recap; To close ..., In conclusion

--- Only say Thank you, don not say thank you for listening

\subsection{Answering Question}

Feel free to stop me anytime if you have a question

I'd be happy to answer this question at the end

Good question, I'll come to back in few minutes

I'm sorry. Are you asking .. ? rephrase questions

I'd like to think about that/ talk to you about that in the break

I understand that you have different point of view

I don't think that you are wright (don not say you are wrong)



\section{Creating Slides}


10-20-30 Rule:

--Do not use more than 10 slides

--Do not speak for more than 20 minutes

--Do not use font smaller than 30 points

-- Use graphics and visual, not text

-- Do not read form the slides

666 Rule

Limit to:
\begin{itemize}
\item{6 words in a bullet point}
\item{6 bullet points on a slide}
\item{6 slides with bullet points in a row}
\end{itemize}

\section{Graphs and Charts}

\subsection{Vocabulary}
the x-axis the y-axis

the solid line, the red line

the dotted line

the broken line

a bar chart/graph

a column

a pie chart

a segment/ share

\subsection{Introduce Visuals}

How have changes in the travel industry \textbf{impacted} travel agents? Let me show

I'm going to show you some data to illustrate how serious this is

This graph \textbf{shows} the change in the number of travel agents in the United State

this chart explains \textbf{how/why/when sales fell}

this diagram \textbf{illustrates the process}

\textbf{This table provides} data for 2016

\textbf{This table list} the names of countries with the top tourist destinations

\textbf{this table gives} information about travel in the last five years

\subsection{Direct Attention}

\textbf{As you can see} the biggest change was from 2000 to 2004, when the industry lost more than 100,000 travel agents.

\textbf{As you can see}

\textbf{Here you can see}

\textbf{I want to point out that}

\textbf{Let me point out that }

\textbf{It's important to notice that} this change began in 2000

\subsection{Show Importance/ Relevance}

\textbf{This means that }

\textbf{This clearly shows that}

\textbf{This is important because}

\textbf{So you can see that .. }

\textbf{This clearly illustrates}

\subsection{Mention the Source}

This graph, \textbf{from the}

\textbf{According to}

\textbf{A study by}

\subsection{Be Specific}

\textbf{People between 18 and 34} are

\textbf{Older people spend}

\textbf{Eighty percent of travelers} ..

\subsection{Support your Main Point}
Explain important:

\textbf{This is important because} ...

\textbf{this explains why} 

Make predictions:

\textbf{This means that} 

\textbf{If this trend continues} ...

\textbf{Based on this information}

Draw a conclusion:

\textbf{It's clear from looking at this chart that}

\textbf{So you can see that}

\textbf{Based on this data}

\textbf{This clearly shows that}


\subsection{Numbers}

Some useful expression:

\begin{itemize}
	\item{Two \textbf{out of} three customer is ...}
	\item{One \textbf{out of every} seven minutes online is Facebook}
	\item{One in four American is }
	\item{Pokhara is less expensive than Zurich}
	\item{Zurich is \textbf{four times as expensive as} Cuzco} 
\end{itemize}

 \subsection{Trends}
 
 a gradual increase in
 
 a decline in
 
 a shape decrease in, drop in, the upstick in
 
 a gradual change, a small change, a huge drop, a very significant increase
 
 --- Prepositions:
 
 in that period
 
 form 40 to 45
 
 increase to 
 
 increase in spending  of 10
 
 increase by 50
 
 
 
 
\end{document}